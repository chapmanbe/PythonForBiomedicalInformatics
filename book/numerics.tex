\chapter{Numeric Types in Python}

\section{Number Systems}

\begin{quote}
    It is not as easy as one might think to say what the word ``number'' means: the more mathematics one learns, the more uses of this word one comes to know, and the more sophisticated one's conception of number becomes....

    The modern view of numbers is that they are best regarded not individually but as parts of  larger wholes, called \emph{number systems;} the distinguishing feature of numbers systems are the arithmetical operations---such as addition, multiplication, subtraction, division, and extraction of roots---that can be performed on them. \emph{The Princeton Companion to Mathematics,} p. 17)
\end{quote}

\subsection{Natural Numbers $\mathbb{N}$}

Our first number system relates to ordering and counting objects. The set of \textbf{Natural Numbers} is denoted by the symbol $\mathbb{N}$. The definition of $\mathbb{N}$ varies. Typically, $\mathbb{N}$ is synonymous with the positive integers, that is $1,2,3,\cdots$. However, $\mathbb{N}$ sometimes is used to denote the non-negative integers: $0,1,2,3,\cdots$; this convention is typically used in set theory.

For natural numbers, addition and multiplication are always defined. That is, whenever we add two natural numbers, the result is \textbf{always} a natural number. Likewise, whenever we multiply two natural numbers, the result is \textbf{always} a natural number.

It is readily apparent, however, that \textbf{subtraction and division are not always defined.} That is, for two natural numbers $a \text{ and } b$ $a-b$ \textbf{might not} be a natural number.

\subsubsection{}

For natural numbers we have the following laws (assuming $a,b,c \in \mathbb{N}$):

\begin{itemize}
    \item Commutative law of addition: $a+b=b+a$
    \item Associative law of addition: $(a+b)+c = a + (b+c)$
    \item Monotonic law of addition: $ \text{ if } a < b \text{ then } a+c<b+c$
    \item Commutative law of multiplication: $ab=ba$
    \item Associative law of multiplication: $(ab)c=a(bc)$
    \item Monotonic law of multiplication: $ \text{ if } a < b \text{ then } ac<bc$
\end{itemize}

\subsection{Integers}
The natural numbers do not include zero or negative numbers, both of which are indispensable, although of relatively recent origin. Writing in the 9th century C.E., \href{https://en.wikipedia.org/wiki/Mu%E1%B8%A5ammad_ibn_M%C5%ABs%C4%81_al-Khw%C4%81rizm%C4%AB}{Al-Khwarizmi} had a symbol for ``nothing'' although it wasn't yet treated as a number.

The set of numbers that do include these are referred to as the \textbf{Integers}. The Integers are denoted by $\mathbb{Z}$, from the German \emph{Zahlen} meaning ``number.''

For integers, the commutative and associative laws all hold, but the monotonic law of multiplication does not hold (as we now include zero and negative numbers).

For integers, subtraction now always results in an integer. So from the example above, $4-6$ results in $-2$ which is an integer.


\subsection{Rational Numbers}

The set of all rational numbers is denoted by $\mathbb{Q}$ for quotients. Quotients arise from making measurements, were we inevitably experience parts of some measurement unit. Rational numbers are formed  by taking the ratio of two whole (integer) numbers.

Mathematically, the system of rational numbers now allows for all operations: additions, subtractions, multiplications, and divisions (except by zero).

\subsection{Real Numbers}

Since the ancient Greeks, it has been know that there are numbers that cannot be expressed as the ratio of two numbers. These numbers are referred to as *irrational numbers*. A famous irrational number is $\sqrt{2}$. The [proof](./sqrt2.ipynb) that $\sqrt{2}$ is irrational, ``is one of the best-known arguments in all of mathematics.'' (\emph{Princeton Companion to Mathematics})


We now need to expand our concept of numbers to what we call the *real* numbers. But real numbes are hard to define.

\begin{quote}
    The real numbers can be thought of as the set of all numbers with a finit or infinite decimal expansion. In the latter case, they are defined not directly but by a process of successive approximations. (Princeton)
\end{quote}

The real numbers are denoted by the symbol $\mathbb{R}$. For real numbers, addition, multiplication, subtraction, division (except by zero), and root taking (except of negative numbers) are defined since they always result in real numbers.

\begin{itemize}
    \item Real numbers include the integers, the rational numbers ($\mathbb{Q}$), and the irrational numbers.
    \item Real numbers can be visualized as a position (distance) along a line
\end{itemize}


%<img src="http://upload.wikimedia.org/wikipedia/commons/thumb/d/d7/Real_number_line.svg/2000px-Real_number_line.svg.png" alt="the real number line" width="500">

\subsubsection{Important (or famous) Real Numbers}

\begin{itemize}
    \item $\sqrt{2}$
    \item $e$ (sometimes called Euler's Number) is an irrational and transcendental (a number that is not the root (solution) to a polynomial equation) defined as
\begin{equation}
e = \lim_{n\to \infty}\left(1+\frac{1}{n}\right)^2=1+\sum_{n=1}^\infty\frac{1}{n!}\approx 2.7182818
\end{equation}

    \item $\pi$: defined as the ratio of a the ratio of the circumference of a circle to its diameter. $\pi$ is an incredibly important number, partially because it is always popping up in unexpected places. Here are a variety of ways that $\pi$ can be defined:
\begin{eqnarray*}
\frac{\pi^2}{6} = \sum_{n=1}^{\infty}\frac{1}{n^2},\\
\frac{\pi}{4} = \left(1-\frac{1}{3}+\frac{1}{5}-\frac{1}{7}+\cdots\right),\\
\pi=\sum_{k=0}^\infty\frac{1}{16^k}\left(\frac{4}{8k+1}-\frac{2}{8k+4}-\frac{1}{8k+5}-\frac{1}{8k+6}\right)
\end{eqnarray*}

    \item $ \phi (\text{ or } \tau )$: the golden ratio $\phi=\frac{1}{22}(1+\sqrt{5})$.
\end{itemize}

\textbf{Note how we were able to take only integers and create an irrational number} ($\pi$)

\subsection{Complex Numbers}

We are all familiar with the solution to the equation
\begin{equation}
x^2=9.
\end{equation}

We are also familiar with the solution to the equation

\begin{equation}
x^2=2.
\end{equation}

However, some of you are probably not familiar with the solution to the equation

\begin{equation}
x^2=-1.
\end{equation}

When the solution to the last equation was first ``discovered'' it was quite controversial, leading to its being named **imaginary**. But before you get worked up and dismiss these numbers, remember that negative numbers were controversial (``false'') as were irrational numbers (e.g. $\sqrt{2}$) when they were discovered. The solution is obtained by defining $i=\sqrt{-1}$.

A \textbf{complex number} $z$ is then defined as

\begin{equation}
z = x+iy
\end{equation}
where $x$ and $y$ are real numbers.

The set of all complex numbers is denoted by $\mathbb{CC}$.

Geometrically, real numbers are the continuous location along a line. Similarly, \textbf{complex numbers} are the continuous location in a plane with the real component describing the $x$-axis and the imaginary component describing the $y$-axis.

%![The complex plane](http://upload.wikimedia.org/wikipedia/commons/7/7d/Gaussian_integer_lattice.png)

\subsubsection{Rules of Complex Arithmetic}

\subsubsection{Conjugate}

Given a complex number $z$, the conjugate of $z$ ($z^*$) is defined as
\begin{equation}
z^*=x-iy
\end{equation}

\subsubsection{Norm}

Multiplying a complex number by its conjugate results in the magnitude squared of the number.
{\bf Addition:}
\begin{equation}
z_1+z_2 = (x_1+x_2)+i(y_1+y_2)
\end{equation}

\subsubsection{Multiplication}
\begin{eqnarray}
z_1*z_2 = x_1x_2+ix_1y_2+iy_1x_2+i^2y_1y_2\\ \nonumber
= (x_1x_2-y_1y_2)+i(x_1y_2+x_2y_1)
\end{eqnarray}

\subsubsection{Division}
\begin{equation}
\frac{z_1}{z_2} = \frac{z_1z^*_2}{z_2z^*_2}=\frac{z_1z^*_2}{||z_2||^2}
\end{equation}

\textbf{The Importance of Data Representation}
Multiplication and division of complex numbers can be simplified by using polar notation. Given $z=x+iy$
\begin{equation}
z=re^{i\theta}
\end{equation}
where $r=\sqrt{x^2+y^2}$ and $\theta=tan^{-1}(\frac{y}{x})$.
Then
\begin{equation}
z_1z_2 = r_1r_2e^{i\theta_1+\theta_2}.
\end{equation}

Despite their name, complex numbers actually simplify life a lot. Complex numbers are complete: any polynomial has complex roots (with real numbers as a subset of complex numbers). Through Euler's equations, trigonometry can be greatly simplified by relating $sin$ and $cos$ to complex functions of $e$.

Complex numbers are convenient for representing any physical quantity that has both a magnitude and a phase (e.g. electrical current in a circuit, pressure waves in the circulation, MR signal).

\section{Basic Numeric Types in Python}

Python has three basic types of numbers:
\begin{itemize}
    \item Integers
    \item Floating Point Numbers
    \item Complex Numbers
\end{itemize}

Why these different types? Aren't numbers just numbers? Numbers arose from two different applications: counting and measuring. (See K. Devlin, \emph{Introduction to Mathematical Thinking} for a brief discussion.)

\begin{enumerate}
    \item \textbf{Counting} led to integers (with negative numbers and zero added along the way)
    \item \textbf{Measuring} led to real numbers
\end{enumerate}

\subsection{Integers: $\mathbb{Z}$, $0,\pm 1, \pm 2, \cdots$}

Python 3.x simplifies the representation of integers (see \href{https://docs.python.org/3/whatsnew/3.0.html#integers}{What's New In Python 3.0}). In Python 2.x integers (limited precision) and long integers (unlimited precision). In 2.x maximum integer value is stored in \textbf{sys.maxsize}
This has a value of 9223372036854775807 on my MacBook Pro.
